% Duda & Hart 章2.6 に対応するオリジナル解説スライド(日本語、数式あり)
\documentclass[dvipdfmx,12pt,notheorems]{beamer}
\usetheme{metropolis}
\usefonttheme{professionalfonts}
\usepackage[utf8]{inputenc}
\usepackage[T1]{fontenc}
\usepackage{xcolor}
\usepackage{graphicx}
\usepackage{amsmath}
\usepackage{mathtools}

% カスタム色コマンド
\newcommand{\red}[1]{\textcolor{red}{#1}}
\newcommand{\green}[1]{\textcolor{green!40!black}{#1}}
\newcommand{\blue}[1]{\textcolor{blue!80!black}{#1}}

% メタ情報
\title{Duda \& Hart \\3.4 LEARNING THE MEAN OF A NORMAL DENSITY}
\subtitle{3.4.2 The Univariate Case:$p(x | \mathcal{X})$ \\3.4.3 The Multivariate Case}
\author{作成者: おおつかたく}
\date{\today}

\begin{document}

\begin{frame}
  \titlepage
\end{frame}

\begin{frame}{目次}
  \tableofcontents
\end{frame}

\section{Purpose and Background}
\begin{frame}{目的と背景}
  \begin{itemize}
    \item 分散が既知で平均が未知の正規分布に対し、ベイズ推定を用いて平均値を学習する。
    \item 未知の $\mu$ を確率変数とみなし、事前知識(事前分布)とデータ(尤度)を組み合わせて推定する。
  \end{itemize}
\end{frame}

\section{3.4.2 1変数の場合: $p(x | \mathcal{X})$}
\begin{frame}{3.4.2 1変数の場合: $p(x | \mathcal{X})$}
  事後密度 $p(\mu | \mathcal{X})$ を得たので、残るは「クラス条件付き」密度 $p(x | \mathcal{X})$ を得ることだけである*。
  式 (14), (15), (19) より、

  $$p(x | \mathcal{X}) = \int p(x | \mu) p(\mu | \mathcal{X}) \, d\mu$$

  \footnotesize
  * 単純化のためにクラスの区別を省略したが、すべてのサンプルは同一のクラス、例えば $\omega_1$ からのものであり、$p(x | \mathcal{X})$ は実際には $p(x | \omega_1, \mathcal{X}_1)$ である。
\end{frame}

\begin{frame}{3.4.2 1変数の場合: $p(x | \mathcal{X})$ 続き}
  \small
  $$p(x | \mathcal{X}) = \int p(x | \mu) p(\mu | \mathcal{X}) \, d\mu$$
  $$= \int \frac{1}{\sqrt{2\pi}\sigma} \exp\left[-\frac{1}{2}\left(\frac{x - \mu}{\sigma}\right)^2\right] \frac{1}{\sqrt{2\pi}\sigma_n} \exp\left[-\frac{1}{2}\left(\frac{\mu - \mu_n}{\sigma_n}\right)^2\right] \, d\mu$$
  $$= \frac{1}{2\pi\sigma\sigma_n} \exp\left[-\frac{1}{2} \frac{(x - \mu_n)^2}{\sigma^2 + \sigma_n^2}\right] f(\sigma, \sigma_n)$$
  ここで、
  $$f(\sigma, \sigma_n) = \int \exp\left[-\frac{1}{2} \frac{\sigma^2 + \sigma_n^2}{\sigma^2 \sigma_n^2} \left(\mu - \frac{\sigma_n^2 x + \sigma^2 \mu_n}{\sigma^2 + \sigma_n^2}\right)^2\right] \, d\mu$$
\end{frame}

\begin{frame}{3.4.2 1変数の場合: $p(x | \mathcal{X})$ 続き}
  つまり、$x$ の関数として見ると、$p(x | \mathcal{X})$ は $\exp\left[-\frac{1}{2}\frac{(x - \mu_n)^2}{\sigma^2 + \sigma_n^2}\right]$ に比例し、したがって $p(x | \mathcal{X})$ は平均 $\mu_n$、分散 $\sigma^2 + \sigma_n^2$ の正規分布に従う。
  $$p(x | \mathcal{X}) \sim N(\mu_n, \sigma^2 + \sigma_n^2) \quad (25)$$
\end{frame}

\begin{frame}{3.4.2 1変数の場合: $p(x | \mathcal{X})$ 続き}
  \begin{itemize}
    \item 言い換えれば、パラメトリックな形式が $p(x | \mu) \sim N(\mu, \sigma^2)$ であることが既知である「クラス条件付き」密度 $p(x | \mathcal{X})$ を得るには、単に $\mu$ を $\mu_n$ に、$\sigma^2$ を $\sigma^2 + \sigma_n^2$ に置き換えるだけでよい。
    \item 事実上、条件付き平均 $\mu_n$ はあたかも真の平均であるかのように扱われ、既知の分散は、平均 $\mu$ に関する正確な知識が欠けていることに起因する $x$ の追加的な不確実性を考慮して増大される。
  \end{itemize}
\end{frame}

\begin{frame}{3.4.2 1変数の場合: $p(x | \mathcal{X})$ 続き}
  \begin{itemize}
    \item これが最終結果である。密度 $p(x | \mathcal{X})$ は求めるクラス条件付き密度 $p(x | \omega_i, \mathcal{X}_i)$ であり、事前確率 $P(\omega_i)$ と合わせることで、ベイズ分類器の設計に必要な確率的情報を与えてくれる。
  \end{itemize}
\end{frame}


\section{3.4.3 多変量の場合}
\begin{frame}{3.4.3 多変量の場合}
  多変量の場合の扱いは、1変数の場合の直接的な一般化である。したがって、証明の概略のみを簡潔に述べる。以前と同様に、以下を仮定する。
  $$p(\vec{x} | \vec{\mu}) \sim N(\vec{\mu}, \mathbf{\Sigma}) \quad (26)$$
  $$p(\vec{\mu}) \sim N(\vec{\mu}_0, \mathbf{\Sigma}_0) \quad (27)$$
  ここで、$\mathbf{\Sigma}, \mathbf{\Sigma}_0, \vec{\mu}_0$ は既知であると仮定する。
\end{frame}

\begin{frame}{3.4.3 多変量の場合 続き}
  $n$ 個の独立したサンプルの集合 $\mathcal{X} = \{\vec{x}_1, \dots, \vec{x}_n\}$ を観測した後、ベイズの規則を用いて以下を得る。
  $$p(\vec{\mu} | \mathcal{X}) = \alpha \prod_{k=1}^n p(\vec{x}_k | \vec{\mu}) p(\vec{\mu})$$
  $$= \alpha' \exp\left[ -\frac{1}{2} \left( \vec{\mu}^t (n\mathbf{\Sigma}^{-1} + \mathbf{\Sigma}_0^{-1})\vec{\mu} - 2\vec{\mu}^t (\mathbf{\Sigma}^{-1} \sum \vec{x}_k + \mathbf{\Sigma}_0^{-1} \vec{\mu}_0) \right) \right]$$
  これは以下の形式を持つ。
  $$p(\vec{\mu} | \mathcal{X}) = \alpha'' \exp\left[ -\frac{1}{2} (\vec{\mu} - \vec{\mu}_n)^t \mathbf{\Sigma}_n^{-1} (\vec{\mu} - \vec{\mu}_n) \right]$$
\end{frame}

\begin{frame}{3.4.3 多変量の場合 続き}
  したがって、$p(\vec{\mu} | \mathcal{X}) \sim N(\vec{\mu}_n, \mathbf{\Sigma}_n)$ となり、ここでもまた再生密度(reproducing density)が得られる。係数を等置することで、式(20)と(21)の類似式を得る。
  $$\mathbf{\Sigma}_n^{-1} = n\mathbf{\Sigma}^{-1} + \mathbf{\Sigma}_0^{-1} \quad (28)$$
  $$\mathbf{\Sigma}_n^{-1} \vec{\mu}_n = n\mathbf{\Sigma}^{-1} \mathbf{m}_n + \mathbf{\Sigma}_0^{-1} \vec{\mu}_0 \quad (29)
  $$ここで $\mathbf{m}_n$ は標本平均である。
  $$\mathbf{m}_n = \frac{1}{n} \sum_{k=1}^n \vec{x}_k \quad (30)$$
\end{frame}

\begin{frame}{3.4.3 多変量の場合 続き}
  $\vec{\mu}_n$ と $\mathbf{\Sigma}_n$ に関するこれらの方程式の解は、以下の行列恒等式の知識によって単純化される。
  $$(\mathbf{A}^{-1} + \mathbf{B}^{-1})^{-1} = \mathbf{A}(\mathbf{A} + \mathbf{B})^{-1}\mathbf{B} = \mathbf{B}(\mathbf{A} + \mathbf{B})^{-1}\mathbf{A}$$
  これは任意の正則な $d \times d$ 行列 $\mathbf{A}$ と $\mathbf{B}$ のペアに対して有効である。少しの操作の後、以下の最終結果を得る。
  $$\vec{\mu}_n = \mathbf{\Sigma}_0 (\mathbf{\Sigma}_0 + \frac{1}{n}\mathbf{\Sigma})^{-1} \mathbf{m}_n + \frac{1}{n}\mathbf{\Sigma} (\mathbf{\Sigma}_0 + \frac{1}{n}\mathbf{\Sigma})^{-1} \vec{\mu}_0 \quad (31)$$
  $$\mathbf{\Sigma}_n = \mathbf{\Sigma}_0 (\mathbf{\Sigma}_0 + \frac{1}{n}\mathbf{\Sigma})^{-1} \frac{1}{n}\mathbf{\Sigma} \quad (32)$$
\end{frame}

\begin{frame}{3.4.3 多変量の場合 続き}
  $p(\vec{x} | \mathcal{X}) \sim N(\vec{\mu}_n, \mathbf{\Sigma} + \mathbf{\Sigma}_n)$ であることの証明は、以前と同様に以下の積分を実行することで得られる。
  $$p(\vec{x} | \mathcal{X}) = \int p(\vec{x} | \vec{\mu}) p(\vec{\mu} | \mathcal{X}) \, d\vec{\mu}$$
\end{frame}

\begin{frame}{3.4.3 多変量の場合 続き}
  \begin{itemize}
    \item しかし、この結果は、$\vec{x}$ が2つの確率変数の和と見なせることに着目すれば、より少ない労力で得られる。
    \item すなわち、$p(\vec{\mu} | \mathcal{X}) \sim N(\vec{\mu}_n, \mathbf{\Sigma}_n)$ に従う確率ベクトル $\vec{\mu}$ と、$p(\vec{y}) \sim N(\vec{0}, \mathbf{\Sigma})$ に従う独立した確率ベクトル $\vec{y}$ の和である。
    \item 独立した正規分布に従う2つのベクトルの和は、再び正規分布に従うベクトルとなり、その平均は平均の和、共分散行列は共分散行列の和となるため、
    $$p(\vec{x} | \mathcal{X}) \sim N(\vec{\mu}_n, \mathbf{\Sigma} + \mathbf{\Sigma}_n)$$
    となり、一般化は完了する。
  \end{itemize}
\end{frame}

\section{Conclusion}
\begin{frame}{まとめ}
  \begin{itemize}
    \item 今回は、分散が既知で平均が未知の正規分布に対し、ベイズ推定を用いて平均値を学習する方法を説明した。
    \item 1変数の場合と多変量の場合の両方で、事後分布とクラス条件付き密度を導出した。
  \end{itemize}
\end{frame}

\end{document}

% 参考文献
% Duda, R. O., & Hart, P. E. (1973). Pattern classification and scene analysis. Wiley.
