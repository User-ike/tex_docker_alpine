\documentclass[dvipdfmx,12pt, notheorems]{beamer}
%%%% 使用するパッケージ類 %%%%%
%%%% テーマ %%%%
%% フレームテーマ %% この中から好きなテーマを選択して
\usetheme{metropolis}
%\usetheme{Boadilla}
%\usetheme{Madrid}
%\usetheme{Antibes}
%\usetheme{Montpellier}
%%%% フォント %%%%
%% フォントテーマ
\usefonttheme{professionalfonts}
%% ゴシック体に強制
\renewcommand{\kanjifamilydefault}{\gtdefault}
%% 文字の色を変更するコマンドを定義
\newcommand{\red}[1]{\textcolor{red}{#1}} %%\red{}内に文字を入力すると赤色になる
\newcommand{\green}[1]{\textcolor{green!40!black}{#1}}
\newcommand{\blue}[1]{\textcolor{blue!80!black}{#1}}
%% 公式を囲う
\usepackage{ascmac}
%% 2段組にする
\usepackage{multicol}
%% 図の挿入
\usepackage{graphicx}
\usepackage[hang,small,bf]{caption}
\usepackage[subrefformat=parens]{subcaption}
\captionsetup{compatibility=false}
\renewcommand{\figurename}{図}
\setlength\abovecaptionskip{0pt}
\makeatletter
%% URLを挿入
\usepackage{url}

%% 以下本文
\begin{document} %% 作成する資料はすべて \begin{document}..\end{document}の中に記述する
\title{\LaTeX, beamerによるプレゼン資料作成サンプル}
\author{名前}
\date{日付}

\begin{frame} %% \begin{frame}..\end{frame} で 1 枚のスライド
  \titlepage %%\titlepage: 上で宣言したtitleやauthor,dateなどをタイトルページとして表示
\end{frame}

\begin{frame}
  \frametitle{目次}  %% フレームのタイトルをつける
  \tableofcontents %% 目次を追加する. なくてもいい
\end{frame}


\begin{frame}{タイトル} %%ここに書いてもタイトルになる
  \frametitle
  超基本的な\LaTeX の使い方 %%ここに記述した文章が表示される.
\end{frame}

\section{1章} %% \sectionでセクションを区切ることができる(tableofcontentsに表示される)
\begin{frame}
  \frametitle{\LaTeX の基本構文}
  文章は\textbackslash begin\{frame\}..\textbackslash end\{frame\}
  のなかに記述することで1枚のスライドに納めることができます.\\
  (ただし文章量が多いとはみ出るので注意)
  %% 文章と文章の間に空白行を挿入すると改段落できる

  文章を任意の場所で改行したい場合は改行したい場所で \\
  "\red{\textbackslash\textbackslash}" %% \\や空白行を挿入しなかった場合は改行されません
  と入力することで段落内改行を行えます.

  階段落するには, 文章と文章の間に空白行を挿入するか,
  文章の最後に"\red{\textbackslash par}"と入力することで改段落できます.\par %% \parを文章の最後につけても改段落できる
\end{frame}

\section{箇条書き}
\begin{frame}{箇条書き}
  箇条書き
  \begin{itemize}
    \item サンプル1
    \item サンプル2
    \item サンプル3
  \end{itemize}
  enumerateで番号をつけれる
  \begin{enumerate} %% enumerateで番号をつけれる
    \item サンプル1
    \item サンプル2
    \item サンプル3
  \end{enumerate}
\end{frame}

\section{数式}
\subsection{基本的な数式} %% subsectionで
\begin{frame}{数式}
  1 + 1 = 3 を\$\$で囲うと\\
  $1 + 1 = 3$のように数式モードで出力される. \par
  式番号を振ったり,中央に表示する場合
  \begin{equation}
    1 + 2 = 3 \label{eq:1}%% \nonumber を式に加えても番号はなくなる
  \end{equation}
  ラベルをつけると\eqref{eq:1}のように参照できる. %%\eqrefが数式の参照コマンド
\end{frame}
\section*{囲い}
\begin{frame}{文字を囲う}
  文章とか定理を囲いたいとき
  \begin{itembox}[c]{基本的な数式} %% []の中は"l:左,c:中央,r:右"
    \begin{equation*}%% * をつけると式番号が振られない
      1 + 1 = 10
    \end{equation*}
  \end{itembox}
\end{frame}

\section*{図}
\begin{frame}{図の挿入}
  \begin{figure}[htbp]
    \includegraphics[scale = 0.18]{images/peterpan_syndrome.png}
    \caption{\green{ピーターパン}} \label{label_2}
  \end{figure}
  図\ref{label_2}のように参照できる %% \refが通常の参照
\end{frame}
\end{document}
